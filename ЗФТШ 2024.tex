% Преамбула
% \documentclass{book}
\documentclass{article}

\usepackage[utf8]{inputenc}
\usepackage[russian]{babel}

\usepackage[T1]{fontenc}
% \usepackage[showframe,a6paper,top=2.5cm]{geometry} 
% \usepackage[a4paper,portrait,left=1.0cm,right=1.0cm,top=0.5cm,bottom=0.5cm]{geometry}
\usepackage[a5paper,portrait,left=1.0cm,right=1.0cm,top=1.0cm,bottom=1.0cm]{geometry}
% \usepackage{gooemacs}
\usepackage{enumitem}
\usepackage{tabularx}
\usepackage{amsmath}
% \usepackage{fancyhdr}

\usepackage{xcolor,graphicx}
\usepackage[font=small,labelfont=bf]{caption}
\usepackage{nopageno}
\newcommand{\pdt}[1]{
	\par
	\begin{minipage}{.95\linewidth}
	\setlength{\parindent}{2em}
	{#1}
	\end{minipage}
	\linebreak
}

\newcommand{\pdtpr}[4]{
	\par
	\begin{minipage}{#3\linewidth}
	\setlength{\parindent}{2em}
	{#1}
	\end{minipage}
	\begin{minipage}{#4\linewidth}
	{\includegraphics[width=\linewidth]{#2}}
	\end{minipage}
	\linebreak
}
\newcommand{\pdtpb}[2]{
	\par
	\begin{minipage}{.95\linewidth}
	\setlength{\parindent}{2em}
	{#1}
	\end{minipage}
	\linebreak
	\begin{minipage}{.95\linewidth}
	{\includegraphics[width=\linewidth]{#2}}
	\end{minipage}
	\linebreak
}



\begin{document}
\newcounter{cn}
\stepcounter{cn}
\pdt{
\Large\textbf{Задача 1.}
 {Если полностью открыть только горячий кран, то ведро воды объёмом $V_1=12\textit{л}$ наполняется за $t_1=80\textit{с}$, а если полностью открыть только холодный кран, то банка объёмом $V_2=5\textit{л}$ наполняется за $t_2=27\textit{с}$.

 За какое время $t$ наполняется кастрюля объемом $V=8,7\textit{л}$, если оба крана открыты полностью? Ответ приведите в секундах, с округлением до целых}
 }
 \linebreak
\pdt{
\Large\textbf{Задача 2.}
 {Неоднородную трубу AB длиной $6,0 \textit{м}$ медленно поднимают с помощью троса. В процессе подъема труба остается горизонтальной.

 На каком расстоянии от края А трубы закреплен трос? Известно что для того, чтобы приподнять край А трубы, лежащей  на горизонтальной поверхности, следует приложить в точке А вертикальную силу $F_A=450\textit{Н}$, а чтобы приподнять край трубы B следует приложить в точке B вертикальную силу $F_B=150\textit{Н}$.

 Ответ приведите в м с округлением до десятых.}
 }
 \linebreak
\pdt{
\Large\textbf{Задача 3.}
 {В океанографическом музее объем кубического аквариума равен $V=512\textit{м}^3$. Весь аквариум заполнен водой. Найдите силу, с которой вода действует на горизонтальное дно аквариума. Ускорение свободного падения $g=10\textit{м/с}^2$, плотность воды $\rho=1000\textit{кг/м}^3$, атмосферное давление $P_0=10^5\textit{Па}$. Ответ приведите в [МН].
 }
 }
 \linebreak
\pdt{
\Large\textbf{Задача 4.}
 {Теплоход прошел первую половину пути с постоянной скоростью $v_1=19\textit{км/ч}$, следующую треть пути - с постоянной скоростью $v_2=14\textit{км/ч}$, оставшиеся 16 км за 1ч. Движение теплохода безостановочное.

 \noindentНайдите среднюю скорость $v$ теплохода за все время движения. Ответ приведитее в [км/ч] с округлением до десятых. 
 }
 }
 \linebreak
\pdt{
\Large\textbf{Задача 5.}
 {Две сплошные фигурки одинаковых размеров и формы сделаны из материалов А, частично - из материала Б. Первая фигурка: треть массы - материал А, остальное - материал Б. Вторая фигурка: треть объёма - материал Б, остальное - материал А. Плотность материала А в 5,2 раза меньше плотности материала Б. Найдите соотношение $\dfrac{M_1}{M_2}$ масс фигурок. Ответ приведите с округлением до десятых.
 }
}
% \pdtpr{
% \Large\textbf{Задача .}
% {Страховка альпиниста закреплена на высоте $h_1=20\text{м}$. Вася массой $m=40\text{кг}$ забрался на высоту 30 м. Какой должна быть жесткость альпинисткой веревки, чтобы при падении Вася не коснулся земли? 
% }}
% {alpin.png}{.6}{.35}
% \linebreak
% \pdtpr{
% \Large\textbf{Задача 3.}
% {Шар закреплен на жестком невесомом стержне длинны $l = 0,45\text{м}$. С какой скоростью $v_0$ должен двигаться шар в нижней точке чтобы он смог сделать полный оборот?
% }}
% {ball.png}{.6}{.35}
% \linebreak
% \pdtpr{
% \Large\textbf{Задача 4.}
% {Вася спрыгнул с балкона слева чтобы Коля смог запрыгнуть на балкон справа. Какова скорость приземления на балкон Коли, если высота балконов 3 метра, Вася весит 35 кг, а Коля 30 кг?
% }}
% {balcon.png}{.6}{.35}
% \linebreak
% \pdtpr{
% \Large\textbf{Задача 5.}
% {
% Папа затащил Васю на горку длинной $l=100\text{м}$ и высотой  $h=50\text{м}$. Какую работу затратил папа на подъем Васи, если сила трения $F_\text{тр}=30\text{Н}$, а масса Васи $m_\text{В}=30\text{кг}$? С какой скоростью будет ехать Вася, когда скатится с горки?
% }}
% {gorka.png}{.6}{.35}
\end{document}