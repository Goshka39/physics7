% Преамбула
% \documentclass{book}
\documentclass{article}

\usepackage[utf8]{inputenc}
\usepackage[russian]{babel}

\usepackage[T1]{fontenc}
% \usepackage[showframe,a6paper,top=2.5cm]{geometry} 
% \usepackage[a4paper,portrait,left=1.0cm,right=1.0cm,top=0.5cm,bottom=0.5cm]{geometry}
\usepackage[a4paper,portrait,left=1.0cm,right=1.0cm,top=1.0cm,bottom=1.0cm]{geometry}
% \usepackage{gooemacs}
\usepackage{enumitem}
\usepackage{tabularx}
\usepackage{amsmath}
% \usepackage{fancyhdr}

\usepackage{xcolor,graphicx}
\usepackage[font=small,labelfont=bf]{caption}
\usepackage{nopageno}
\newcommand{\pdt}[1]{
	\par
	\begin{minipage}{.95\linewidth}
	\setlength{\parindent}{2em}
	{#1}
	\end{minipage}
	\linebreak
}

\newcommand{\pdtpr}[4]{
	\par
	\begin{minipage}{#3\linewidth}
	\setlength{\parindent}{2em}
	{#1}
	\end{minipage}
	\begin{minipage}{#4\linewidth}
	{\includegraphics[width=\linewidth]{#2}}
	\end{minipage}
	\linebreak
}
\newcommand{\pdtpb}[2]{
	\par
	\begin{minipage}{.95\linewidth}
	\setlength{\parindent}{2em}
	{#1}
	\end{minipage}
	\linebreak
	\begin{minipage}{.95\linewidth}
	{\includegraphics[width=\linewidth]{#2}}
	\end{minipage}
	\linebreak
}


\begin{document}
\section*{Вариант 2}
\noindent
\pdtpr{
% \parindent
\Large\textbf{Задача 1.}
 {Температура тела здорового человека равна +36,6°C — такую температуру называют нормальной. На рисунке изображены три термометра. Чему равна цена деления того термометра, который подойдет для измерения температуры тела с необходимой точностью?}
}
{temp.png}{.55}{.4}
% \linebreak
\pdt{
% \parindent
\Large\textbf{Задача 2.}
{Космонавт на орбитальной космической станции решил утром сделать зарядку так, как он всегда делал её дома на Земле: встать, попрыгать, понаклоняться, поприседать, поднять с пола гантели. Но выполнить эти простые упражнения он не смог. Из-за какого физического явления у него ничего не получилось? В чём оно состоит?}
}
\linebreak
\pdt{
\Large\textbf{Задача 3.}
{Для приготовления домашнего майонеза Маше нужно 276г оливкового масла. К сожалению, у неё под рукой нет весов, но зато в кухонном шкафу есть мерный стаканчик для жидкостей. Маша нашла в учебнике физики таблицу, в которой было указано, что плотность оливкового масла равно $0,920\text{г/см}^{3}$. Какой объём масла нужно отмерить Маше? Ответ запишите в мл.}
}
\linebreak
\pdtpr
{\Large\textbf{Задача 4.}
{Вася тренируется перед школьными соревнованиями — выполняет упражнение «челночный бег». При помощи графика зависимости координаты Васи от времени определите путь, пройденный мальчиком за один забег длительностью 16 секунд. Ответ запишите в метрах.}
}{cheln.png}{.6}{.35}
\linebreak
\pdt{
\Large\textbf{Задача 5.}
{В стакан, имеющий форму цилиндра с площадью дна $18 \text{см}^{2}$, налита вода. Вася заметил, что если положить в этот стакан 24 одинаковых скрепок, то уровень воды поднимется на 0,2 см. Чему равен объём одной скрепки? Ответ дайте в $\text{см}^{3}$.}
}
\linebreak
\pdt{
\Large\textbf{Задача 6.}
 {Для закачивания бензина в подземную цистерну на автозаправочной станции используется насос производительностью 80 литров в минуту. Какое время понадобится для заполнения при помощи этого насоса прямоугольной цистерны размерами 3 м × 2,4 м × 2,4 м? Ответ дайте в минутах.}
 }
\linebreak

\pdtpr{
\Large\textbf{Задача 7.}
{Соня решила проверить — справедлив ли закон Гука для резинки для волос. В кабинете физики она взяла набор одинаковых грузиков массой по 50 г каждый и стала подвешивать их к резинке. Определите, выполняется ли закон Гука для изучаемой резинки? Ответ кратко поясните.}}
{guk2.png}{.5}{.45}
\linebreak
\pdt{
\Large\textbf{Задача 8.}
{Спортсмен, занимающийся дайвингом, погрузился в воду на глубину 65 метров. Определите, во сколько раз отличается давление, которое испытывает на себе спортсмен на этой глубине, от давления, испытываемого им на поверхности воды, если давление, создаваемое десятью метрами водяного столба, эквивалентно атмосферному давлению.
}}

\pdt{
\Large\textbf{Задача 9.}
{Средняя плотность карандаша, состоящего из грифеля и деревянной оболочки, равна 700 $\text{кг/м}^3$. Известно, что объём всего карандаша 6 см3, а масса грифеля 1,2 г.

1) Чему равна средняя плотность карандаша, выраженная в $\text{г/cм}^3$?

2) Найдите массу деревянной оболочки.}}

\pdtpr{
\Large\textbf{Задача 10.}
{На рисунке изображены графики зависимостей пути, пройденного грузовым теплоходом вдоль берега, от времени при движении по течению реки и против её течения.

1)Определите скорость теплохода при движении по течению реки.

2)Определите скорость теплохода при движении против течения реки.

3)Какой путь сможет пройти этот теплоход за 30 мин при движении по озеру?
}
}{parahod.png}{.55}{.4}

\pdtpb{\Large\textbf{Задача 11.}
{Вдоль стоящего на станции пассажирского поезда идёт обходчик. Он резко ударяет молотком по оси каждого колеса и затем на мгновение прикладывает к ней руку. Пассажир Иван Иванович заметил, что вдоль всего состава обходчик проходит за 6 минут, делая при этом 48 ударов. Пользуясь чертежом вагона, оцените:

1) сколько вагонов в поезде?

2) с какой средней скоростью идёт обходчик?

3) чему равен минимальный интервал времени между слышимыми ударами?}
}
{vagon.png}
% \par
% \Large\textbf{Задача 2.}
% { Насколько давление воды на глубине $10\hspace{2px}\textit{м}$ больше чем на глубине $1\textit{м}$?}
% \par
% \Large\textbf{Задача 3.}
% { В ёмкости, наполененной нефтью до верха, на расстоянии $140\hspace{2px}\textit{см}$ от крышки имеется кран. Определите давление на кран.}
% \par
% \Large\textbf{Задача 4.}
% { Сосуд имеющий форму показанную на рисунке заполнен водой (рис.1). Рассчитайте давление на каждую из 4 пробок.}
% \par
% \Large\textbf{Задача 5.}
% { В сосуд налиты машинное масло и вода слоями в $3\hspace{2px}\textit{см}$ каждая. Определите давление в $2\hspace{2px}\textit{см}$ от дна сосуда. Плотность масла принять за $900\hspace{2px}\dfrac{\textit{кг}}{\textit{м}^3}$, плотность воды за $1000\hspace{2px}\dfrac{\textit{кг}}{\textit{м}^3}$. Считайте что жидкости не смешиваются.}
% \par
% \Large\textbf{Задача 6.}
% { В двух цилиндрических сосудах налита вода (рис.2). В каком сосуде давление  на дно больше и насколько, если $h_1=48\hspace{2px}\textit{cм}$, а $h_2=14\hspace{2px}\textit{cм}$? Какой уровень воды установится в сосудах, после того как кран откроют, если диаметр правого сосуда в 4 раза больше левого?}

% \par
% \begin{minipage}{0.55\linewidth}
% \includegraphics[width=\linewidth]{d4_grey.png}
% \captionof{figure}{Задача 4}
% \end{minipage}
% \begin{minipage}{0.3\linewidth}
% \includegraphics[width=\linewidth]{d6_grey.png}
% \captionof{figure}{Задача 6}
% \end{minipage}


% \includegraphics[width=.3\textwidth]{}
% \includegraphics[width=.3\textwidth]{d6.png}
\end{document}

