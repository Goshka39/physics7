% Преамбула
% \documentclass{book}
\documentclass{article}

\usepackage[utf8]{inputenc}
\usepackage[russian]{babel}

\usepackage[T1]{fontenc}
% \usepackage[showframe,a6paper,top=2.5cm]{geometry} 
% \usepackage[a4paper,portrait,left=1.0cm,right=1.0cm,top=0.5cm,bottom=0.5cm]{geometry}
\usepackage[a4paper,portrait,left=1.0cm,right=1.0cm,top=1.0cm,bottom=1.0cm]{geometry}
% \usepackage{gooemacs}
\usepackage{enumitem}
\usepackage{tabularx}
\usepackage{amsmath}
% \usepackage{fancyhdr}

\usepackage{xcolor,graphicx}
\usepackage[font=small,labelfont=bf]{caption}
\usepackage{nopageno}
\newcommand{\pdt}[1]{
	\par
	\begin{minipage}{.95\linewidth}
	\setlength{\parindent}{2em}
	{#1}
	\end{minipage}
	\linebreak
}

\newcommand{\pdtpr}[4]{
	\par
	\begin{minipage}{#3\linewidth}
	\setlength{\parindent}{2em}
	{#1}
	\end{minipage}
	\begin{minipage}{#4\linewidth}
	{\includegraphics[width=\linewidth]{#2}}
	\end{minipage}
	\linebreak
}
\newcommand{\pdtpb}[2]{
	\par
	\begin{minipage}{.95\linewidth}
	\setlength{\parindent}{2em}
	{#1}
	\end{minipage}
	\linebreak
	\begin{minipage}{.95\linewidth}
	{\includegraphics[width=\linewidth]{#2}}
	\end{minipage}
	\linebreak
}


\begin{document}
\section*{Вариант 1}
\noindent
\pdtpr{
% \parindent
\Large\textbf{Задача 1.}
 {Давление в системе холодного водоснабжения многоэтажных домов по правилам не должно превышать $6,5\hspace{2px}\textit{бар}$. Вася посмотрел на манометр, присоединённый к трубе подачи холодной воды, шкала которого показывает давление в бар. На сколько давление воды в трубе меньше максимально допустимого? Ответ запишите в бар.}
}
{bar.png}{.75}{.2}
% \linebreak
\pdt{
% \parindent
\Large\textbf{Задача 2.}
{Если налить в одну банку жидкий мёд и воду, не перемешивая их, то мёд опустится вниз, а вода останется сверху над мёдом. Назовите физическую характеристику вещества, благодаря которой мёд погружается в воду. Запишите формулу, при помощи которой можно вычислить эту характеристику, и назовите все входящие в эту формулу обозначения.}
}
\linebreak
\pdt{
\Large\textbf{Задача 3.}
{Однажды вечером Вася решил выйти на прогулку. Он обошёл весь парк за две пятых часа, двигаясь с постоянной скоростью, равной 3,5 км/ч. Сколько километров прошёл Вася по парку? Ответ запишите в километрах.}
}
\linebreak
\pdtpr
{\Large\textbf{Задача 4.}
{Вася вместе с семьёй отправился в путешествие на автомобиле. Во время поездки они проезжали несколько населённых пунктов, в которых приходилось ехать медленнее, чем на трассе. По графику зависимости скорости машины от времени определите, сколько всего времени машина ехала по населённым пунктам, если в населённом пункте нельзя ехать со скоростью, превышающей 60км/ч. Володин папа, который вёл машину, не нарушал правила дорожного движения. Ответ запишите в часах.}
}{vtgraph.png}{.6}{.35}
\linebreak
\pdt{
\Large\textbf{Задача 5.}
	{Васе стало интересно, чему примерно равен объём картофелины среднего размера. Он попросил у учителя физики 10 цилиндров объёмом 60 мл каждый и положил их в кастрюлю, после чего налил туда воду почти доверху. Затем Илья вынул из кастрюли все цилиндры и начал класть в неё картофелины. Оказалось, что после погружения восьми картофелин уровень воды в кастрюле вернулся к уровню, который был до вынимания цилиндров. Оцените объём одной картофелины, считая, что все они были примерно одинаковыми. Ответ дайте в миллилитрах.}
}
\linebreak
\pdt{
\Large\textbf{Задача 6.}
 {Направляясь на день рождения к Соне, Вася купил в магазине связку из 14 воздушных шаров. Но, выйдя на улицу, он обнаружил, что из-за низкой температуры на улице объём шариков уменьшился. Вася предположил, что плотность газа в шариках при охлаждении увеличилась в 1,1 раза. Определите, на сколько литров уменьшился при этом суммарный объём шаров, если предположение Васи верно, а исходный объём одного шарика был равен 3л? Ответ дайте в литрах.}
 }
\linebreak

\pdtpr{
\Large\textbf{Задача 7.}
{Соня решила проверить — справедлив ли закон Гука для резинки для волос. В кабинете физики она взяла набор одинаковых грузиков массой по 50 г каждый и стала подвешивать их к резинке. Определите, выполняется ли закон Гука для изучаемой резинки? Ответ кратко поясните.}}
{guk1.png}{.5}{.45}
\linebreak
\pdtpr{
\Large\textbf{Задача 8.}
{Вася решил попробовать определить внутренний объём надутого воздушного шарика — наполнить его водой и измерить объём этой воды. Выяснилось, что надуть шарик водой не так-то просто, поскольку он не растягивается под её весом. Поэтому Вася начал заливать в шарик воду через вертикальную трубку, как показано на рисунке. Известно, что минимальное дополнительное давление, которое нужно создать для надувания шарика, составляет 8 кПа. Какой минимальной длины трубку надо взять Васе для того, чтобы исполнить свой план? Плотность воды $1000 \text{кг/м}^{3}$. Ответ дайте в метрах.
}}
{ballvol.png}{.8}{.15}

\pdt{
\Large\textbf{Задача 9.}
{Некоторые люди любят пить ароматизированный травяной чай и используют для его приготовления разведённую в воде густую вытяжку из душицы и мать-и-мачехи. Плотность травяной вытяжки $1,24 \text{г/см}^{3}$, плотность воды $1 \text{г/см}^{3}$. Для приготовления раствора смешали одинаковые объёмы воды и травяной вытяжки.
1) Определите массу использованной травяной вытяжки, если её объём равен 100 мл.
2) Найдите плотность полученного раствора, если его объём равен сумме объёмов исходных компонентов.}}

\pdt{
\Large\textbf{Задача 10.}
{Очень сложно путешествовать по тайге в зимнюю пору, когда выпало много снега. Охотник сначала треть пути прошёл за 1/4 всего времени движения, далее одну шестую часть пути он преодолел за 1/4 всего времени. Последний участок пути был пройден охотником со средней скоростью 1,2м/с.
1)Какую часть всего пути охотник шёл со скоростью 1,2м/с? Ответ дайте в виде несократимой дроби.
2)Какую часть всего времени охотник шёл со скоростью 1,2м/с? Ответ дайте в виде несократимой дроби.
3)Найдите среднюю скорость охотника на всём пути.}
}

\pdtpb{\Large\textbf{Задача 11.}
{Электрокардиография (ЭКГ) — один из важных методов исследования работы сердца. Принцип работы аппарата ЭКГ таков: сигнал с датчиков, прикреплённых на различные участки тела, записывается на движущуюся с постоянной скоростью клетчатую бумажную ленту. Длина стороны одной клеточки на бумаге 1 мм (такую бумагу часто называют «миллиметровка»). Обычно на электрокардиограмме можно выделить пять соответствующих сердечному циклу зубцов: P, Q, R, S, T (см. схему). По виду кривой можно судить о состоянии пациента.
Ниже представлена фотография фрагмента электрокардиограммы (одновременно записывался сигнал с трёх датчиков) и увеличенный снимок одного из сердечных сигналов. Скорость движения ленты при проведении этого исследования составляла 25 мм/с. Определите:
1)частоту пульса пациента (количество ударов в минуту);
2)продолжительность интервала PR;
3)длительность промежутка времени, соответствующего приведённому фрагменту.}
}
{ECG.png}
% \par
% \Large\textbf{Задача 2.}
% { Насколько давление воды на глубине $10\hspace{2px}\textit{м}$ больше чем на глубине $1\textit{м}$?}
% \par
% \Large\textbf{Задача 3.}
% { В ёмкости, наполененной нефтью до верха, на расстоянии $140\hspace{2px}\textit{см}$ от крышки имеется кран. Определите давление на кран.}
% \par
% \Large\textbf{Задача 4.}
% { Сосуд имеющий форму показанную на рисунке заполнен водой (рис.1). Рассчитайте давление на каждую из 4 пробок.}
% \par
% \Large\textbf{Задача 5.}
% { В сосуд налиты машинное масло и вода слоями в $3\hspace{2px}\textit{см}$ каждая. Определите давление в $2\hspace{2px}\textit{см}$ от дна сосуда. Плотность масла принять за $900\hspace{2px}\dfrac{\textit{кг}}{\textit{м}^3}$, плотность воды за $1000\hspace{2px}\dfrac{\textit{кг}}{\textit{м}^3}$. Считайте что жидкости не смешиваются.}
% \par
% \Large\textbf{Задача 6.}
% { В двух цилиндрических сосудах налита вода (рис.2). В каком сосуде давление  на дно больше и насколько, если $h_1=48\hspace{2px}\textit{cм}$, а $h_2=14\hspace{2px}\textit{cм}$? Какой уровень воды установится в сосудах, после того как кран откроют, если диаметр правого сосуда в 4 раза больше левого?}

% \par
% \begin{minipage}{0.55\linewidth}
% \includegraphics[width=\linewidth]{d4_grey.png}
% \captionof{figure}{Задача 4}
% \end{minipage}
% \begin{minipage}{0.3\linewidth}
% \includegraphics[width=\linewidth]{d6_grey.png}
% \captionof{figure}{Задача 6}
% \end{minipage}


% \includegraphics[width=.3\textwidth]{}
% \includegraphics[width=.3\textwidth]{d6.png}
\end{document}

