% Преамбула
% \documentclass{book}
\documentclass{article}

\usepackage[utf8]{inputenc}
\usepackage[russian]{babel}

\usepackage[T1]{fontenc}
% \usepackage[showframe,a6paper,top=2.5cm]{geometry} 
% \usepackage[a4paper,portrait,left=1.0cm,right=1.0cm,top=0.5cm,bottom=0.5cm]{geometry}
\usepackage[a5paper,portrait,left=1.0cm,right=1.0cm,top=1.0cm,bottom=1.0cm]{geometry}
% \usepackage{gooemacs}
\usepackage{enumitem}
\usepackage{tabularx}
\usepackage{amsmath}
% \usepackage{fancyhdr}

\usepackage{xcolor,graphicx}
\usepackage[font=small,labelfont=bf]{caption}
\usepackage{nopageno}
\newcommand{\pdt}[1]{
	\par
	\begin{minipage}{.95\linewidth}
	\setlength{\parindent}{2em}
	{#1}
	\end{minipage}
	\linebreak
}

\newcommand{\pdtpr}[4]{
	\par
	\begin{minipage}{#3\linewidth}
	\setlength{\parindent}{2em}
	{#1}
	\end{minipage}
	\begin{minipage}{#4\linewidth}
	{\includegraphics[width=\linewidth]{#2}}
	\end{minipage}
	\linebreak
}
\newcommand{\pdtpb}[2]{
	\par
	\begin{minipage}{.95\linewidth}
	\setlength{\parindent}{2em}
	{#1}
	\end{minipage}
	\linebreak
	\begin{minipage}{.95\linewidth}
	{\includegraphics[width=\linewidth]{#2}}
	\end{minipage}
	\linebreak
}



\begin{document}
\newcounter{cn}
\stepcounter{cn}
\pdt{
\Large\textbf{Задача 1.}
 {Вася бросает мячики массой 30 г в мишень которую держит Коля на высоте 0,8 м выше Васи.  Считая что стартовая скорость мячика 5 м/с, определите минимальную скорость мячика при попадании в мишень. Какую работу нужно совершить Коле чтобы поймать 3 мячика?}
 }
 \linebreak
\pdtpr{
\Large\textbf{Задача 2.}
{Страховка альпиниста закреплена на высоте $h_1=20\text{м}$. Вася массой $m=40\text{кг}$ забрался на высоту 30 м. Какой должна быть жесткость альпинисткой веревки, чтобы при падении Вася не коснулся земли? 
}}
{alpin.png}{.6}{.35}
\linebreak
\pdtpr{
\Large\textbf{Задача 3.}
{Шар закреплен на жестком невесомом стержне длинны $l = 0,45\text{м}$. С какой скоростью $v_0$ должен двигаться шар в нижней точке чтобы он смог сделать полный оборот?
}}
{ball.png}{.6}{.35}
\linebreak
\pdtpr{
\Large\textbf{Задача 4.}
{Вася спрыгнул с балкона слева чтобы Коля смог запрыгнуть на балкон справа. Какова скорость приземления на балкон Коли, если высота балконов 3 метра, Вася весит 35 кг, а Коля 30 кг?
}}
{balcon.png}{.6}{.35}
\linebreak
\pdtpr{
\Large\textbf{Задача 5.}
{
Папа затащил Васю на горку длинной $l=100\text{м}$ и высотой  $h=50\text{м}$. Какую работу затратил папа на подъем Васи, если сила трения $F_\text{тр}=30\text{Н}$, а масса Васи $m_\text{В}=30\text{кг}$? С какой скоростью будет ехать Вася, когда скатится с горки?
}}
{gorka.png}{.6}{.35}
\end{document}